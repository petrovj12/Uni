\documentclass[11pt, a4paper]{article}

\usepackage{amsthm,amsmath, amsfonts}
%\usepackage[ngerman]{babel}
\usepackage{tcolorbox}
\usepackage{hyperref, cleveref}
\usepackage{xcolor, thmtools}
\usepackage{enumitem}
%\usepackage{marginnote}
\usepackage{geometry}

% Hello World

\newcounter{aufgabe}

\geometry{
	left=1in,
	right=2.5in,
	top=2cm,
	bottom=3cm,
	marginparwidth=3cm,
	marginparsep=0.8cm,
	a4paper
}

%----Fonts----
\usepackage{libertinus}
\usepackage[libertine]{newtxmath}

%--Colors--
\definecolor{astral}{RGB}{3,57,108}
\definecolor{dblue}{RGB}{1,31,75}


%%%%%%%%%%%%%
%% General %%
%%%%%%%%%%%%%

\newcommand{\C}{\mathbb{C}}
\newcommand{\R}{\mathbb{R}}
\newcommand{\Q}{\mathbb{Q}}
\newcommand{\N}{\mathbb{N}}
\newcommand{\Z}{\mathbb{Z}}
\newcommand{\HH}{\mathbb{H}}
\newcommand{\ohne}{\setminus}
\newcommand{\card}[1]{\operatorname{\#}\left(#1 \right)}
\renewcommand{\iff}{\Leftrightarrow}
\renewcommand{\implies}{\Rightarrow}
\renewcommand{\tilde}{\widetilde}
\renewcommand{\bar}{\overline}
\newcommand{\aufz}[2]{#1_1, \dots, #1_{#2}}
\newcommand{\calp}{\mathcal{P}}
\newcommand{\vphi}{\varphi}
\newcommand{\norm}[1]{\lVert #1 \rVert}
\newcommand{\abs}[1]{\lvert #1 \rvert}
\DeclareMathOperator{\re}{Re}
\DeclareMathOperator{\imag}{Im}

%%%%%%%%%%%%%%%%%%%%%
%% Category Theory %%
%%%%%%%%%%%%%%%%%%%%%

\DeclareMathOperator{\Ob}{Ob}
\newcommand{\calc}{\mathcal{C}}
\DeclareMathOperator{\Set}{\mathbf{Set}}
\DeclareMathOperator{\CTop}{\mathbf{Top}}
\DeclareMathOperator{\CVec}{\mathbf{Vec}}
\DeclareMathOperator{\Grp}{\mathbf{Grp}}

%%%%%%%%%%%%%
%% Algebra %%
%%%%%%%%%%%%%

\DeclareMathOperator{\Grad}{Grad}
\DeclareMathOperator{\minpol}{minpol}
\DeclareMathOperator{\charpol}{charpol}
\DeclareMathOperator{\ord}{ord}
\DeclareMathOperator{\im}{im}
\DeclareMathOperator{\Bild}{Bild}
\DeclareMathOperator{\ggT}{ggT}
\DeclareMathOperator{\kgv}{kgV}
\DeclareMathOperator{\Gal}{Gal}
\DeclareMathOperator{\Aut}{Aut}
\DeclareMathOperator{\Bij}{Bij}
\DeclareMathOperator{\Abb}{Abb}
\DeclareMathOperator{\Hom}{Hom}
\DeclareMathOperator{\Mor}{Mor}
\DeclareMathOperator{\Mod}{Mod}
\DeclareMathOperator{\sgn}{sgn}
\DeclareMathOperator{\F}{\mathbb{F}}
\DeclareMathOperator{\id}{id}
\DeclareMathOperator{\Ker}{Ker}
\DeclareMathOperator{\Cl}{Cl}
\DeclareMathOperator{\Span}{Span}
\DeclareMathOperator{\Spec}{Spec}
\DeclareMathOperator{\Lok}{Lok}
\DeclareMathOperator{\OK}{\mathcal{O}_K}
\DeclareMathOperator{\OL}{\mathcal{O}_K}
\DeclareMathOperator{\einmat}{\mathbb{E}}
\DeclareMathOperator{\GL}{GL}
\DeclareMathOperator{\SL}{SL}
\DeclareMathOperator{\Orth}{O}
\DeclareMathOperator{\SO}{SO}
\DeclareMathOperator{\rank}{rank}
\newcommand{\pp}{\mathfrak{p}}
\renewcommand{\aa}{\mathfrak{a}}
\newcommand{\bb}{\mathfrak{b}}
\newcommand{\qq}{\mathfrak{q}}
\newcommand{\cc}{\mathfrak{c}}
\newcommand{\mm}{\mathfrak{m}}
\newcommand{\zpz}{\Z / p\Z}
\newcommand{\zmz}{\Z / m\Z}
\newcommand{\znz}{\Z / n\Z}
\newcommand{\zNz}{\Z / N\Z}
\newcommand{\calt}{\mathcal{T}}
\newcommand{\caln}{\mathcal{N}}
\newcommand{\Zi}{\Z[i]}
\newcommand{\Qi}{\Q[i]}
\newcommand{\zmod}[1]{\Z / #1\Z}
\newcommand{\kxn}[1]{k[x_1, \dots, x_{#1}]}
\newcommand{\kx}{k[x]}
\newcommand{\kyn}[1]{k[y_1, \dots, y_{#1}]}
\newcommand{\ky}{k[y]}
\newcommand{\gen}[1]{\langle #1 \rangle}
\newcommand{\epim}{\twoheadrightarrow}
\newcommand{\monom}{\hookrightarrow}
\newcommand{\nsub}{\vartriangleright}
\newcommand{\nsubl}{\vartriangleleft}

%%%%%%%%%%%%%%
%% Analysis %%
%%%%%%%%%%%%%%

\newcommand{\eps}{\varepsilon}
\newcommand{\veps}{\epsilon}
\DeclareMathOperator{\holo}{\mathcal{O}}
\DeclareMathOperator{\vol}{vol}

%%%%%%%%%%%%%%
%% Theorems %%
%%%%%%%%%%%%%%

%% Boxed Theorems %%
\tcbuselibrary{theorems}
\newtcbtheorem[number within=section, crefname={\mathrm{Satz}}{Satz}]{satz}{Satz}
{before skip=2em, after skip=2em, colback=red!5!white, colframe=red!50!black}{S}
\newtcbtheorem[number within=section, crefname={\mathrm{Theorem}}{Theorem}]{thm}{Theorem}
{before skip=2em, after skip=2em, colback=red!5!white, colframe=red!50!black}{S}
\newtcbtheorem[number within=section, crefname={\mathrm{Korollar}}{Korollar}]{kor}
{Korollar}{before skip=2em, after skip=2em}{K}
\newtcbtheorem[number within=section, crefname={\mathrm{Corollary}}{Corollary}]{cor}
{Corollary}{before skip=2em, after skip=2em}{C}
\newtcbtheorem[number within=section, crefname={\mathrm{Definition}}{Definition}]{Def}
{Definition}{before skip=2em, after skip=2em, colback=green!5!white, colframe=green!50!black}{D}
\newtcbtheorem[number within=section, crefname={\mathrm{Hilfssatz}}{Hilfssatz}]{hilfs}
{Hilfssatz}{before skip=2em, after skip=2em}{HS}
\newtcbtheorem[number within=section, crefname={\mathrm{Lemma}}{Lemma}]{Lemma}{Lemma}
{before skip=2em, after skip=2em}{L}
\newtcbtheorem[number within=section, crefname={\mathrm{Folgerung}}{Folgerung}]{folg}
{Folgerung}{before skip=2em, after skip=2em}{FOLG}
\newtcbtheorem[number within=section, crefname={\mathrm{Aufgabe}}{Aufgabe}]{auf}{Aufgabe}{before skip=2em, after skip=2em, colback=blue!5!white, colframe=blue!50!black}{A}
\newtcbtheorem[number within=section, crefname={\mathrm{Proposition}}{Proposition}]
{prop}{Proposition}{before skip=2em, after skip=2em, colback=yellow!5!white, colframe=brown!50!black}{P}

%% Normal Theorems %%
\declaretheoremstyle[
	headfont=\normalfont\bfseries,
	spaceabove=1em,
	spacebelow=1em
]{beisp}
\declaretheoremstyle[
	headfont=\color{purple}\normalfont\bfseries,
	spaceabove=1em,
	spacebelow=1em
]{beme}
\declaretheoremstyle[
	headfont=\color{astral}\normalfont\bfseries,
	spaceabove=1em,
	spacebelow=1em
]{aufg}
\declaretheorem[
	style=beisp,
	name=Beispiel,
	numberwithin=section
]{bsp}
\declaretheorem[
	style=beisp,
	name=Example,
	numberwithin=section
]{ex}
\declaretheorem[
	style=beisp,
	name=Remark,
	numberwithin=section
]{rem}
\declaretheorem[
	style=beme,
	name=Bemerkung,
	numberwithin=section
]{bem}
\declaretheorem[
	style=aufg,
	name=Aufgabe,
]{Aufgabe}




\usepackage{float}
\usepackage{fancyhdr}

\begin{document}

%\fancyhead[L]{Logik}
\setlength{\headheight}{14pt}

\title{Logik}
\author{Jovan Petrov}
\maketitle

\pagestyle{fancy}

\renewcommand{\AA}{\mathfrak{A}}
\newcommand{\BB}{\mathfrak{B}}
\newcommand{\CC}{\mathfrak{C}}

\newcommand{\Bund}{(\text{B}\land)}
\newcommand{\Boder}{(\text{B}\lor)}
\newcommand{\Bsub}{(\text{B}\rightarrow)}

\newcommand{\Eund}{(\text{E}\land)}
\newcommand{\Eoder}{(\text{E}\lor)}
\newcommand{\Esub}{(\text{E}\rightarrow)}

\newcommand{\Prae}[1]{(\text{P}#1)}
\newcommand{\An}[1]{(\text{A}#1)}


\section{Kalkül des natürlichen Schließens}

\subsection{Konjunktion}

\begin{regel}{Konjunktionseinführung}{}
	$\Eund$: $\AA_L, \BB_{L'} \implies (\AA \land \BB)_{L \cup L'}$		
\end{regel}

\begin{regel}{Konjunktionsbeseitigung $(1)$}{}
	$\Bund_1$: $(\AA \land \BB)_{L} \implies \AA_L$		
\end{regel}

\begin{regel}{Konjunktionsbeseitigung $(1)$}{}
	$\Bund_2$: $(\AA \land \BB)_{L} \implies \BB_L$		
\end{regel}

\begin{bsp}(Kommutativität der Konjunktion) \ 
	Wir zeigen, dass $A \land B \models B \land A$
	\begin{figure}[h]
	\centering
	
	
	\begin{tabular}{l r | l l l}
		$1$ & $A \land B$ &  $\Prae{1}$ & &\\
		$2$ & $A$ &  $\Bund_1$ & $1$ & $(1)$\\
		$3$ & $B$ &  $\Bund_2$ & $1$ & $(1)$\\
		$4$ & $B \land A$ &  $\Eund$ & $2,3$ & $(1)$\\
	\end{tabular}

	\end{figure}

\end{bsp}

\begin{bsp}(Assoziativität der Konjunktion) \ 
	Wir zeigen, dass $A \land (B \land C) \models (A \land B) \land C$
	\begin{figure}[h]
	\centering
	
	
	\begin{tabular}{l r | l l l}
		$1$ & $A \land (B \land C)$ &  $\Prae{1}$ & &\\
		$2$ & $A$ &  $\Bund_1$ & $1$ & $(1)$\\
		$3$ & $B \land C$ &  $\Bund_2$ & $1$ & $(1)$\\
		$4$ & $B$ &  $\Bund_1$ & $3$ & $(1)$\\
		$5$ & $C$ &  $\Bund_2$ & $3$ & $(1)$\\
		$6$ & $A \land B$ &  $\Eund$ & $2,4$ & $(1)$\\
		$7$ & $(A \land B) \land C$ &  $\Eund$ & $6,5$ & $(1)$\\
	\end{tabular}

	\end{figure}
\end{bsp}

\subsection{Subjunktion}

\begin{regel}{Subjunktionseinführung}{}
	$\Esub$: $[\AA_i] \dots \BB_L \implies (\AA \rightarrow \BB)_{L \ohne \{i\} }$		
\end{regel}

\begin{regel}{Subjunktionsbeseitigung/ Modus ponens}{}
	$\Bsub$: $(\AA \rightarrow \BB)_L , \AA_{L'} \implies \BB_{L \cup L'}$		
\end{regel}

\begin{bsp}
	Wir zeigen, dass $A \rightarrow (B \land C) \models A \rightarrow B$
	\begin{figure}[h]
	\centering
	
	
	\begin{tabular}{l r | l l l}
		$1$ & $A \rightarrow (B \land C)$ &  $\Prae{1}$ & &\\
		$2$ & $A$ &  $\An{2}$ & &\\
		$3$ & $B \land C$ &  $\Bsub$ & 1,2& (1,2)\\
		$4$ & $B$ &  $\Bund$ & 3& (1,2)\\
		$5$ & $A \rightarrow B$ &  $\Esub$ & 4& (1)\\
	\end{tabular}

	\end{figure}
\end{bsp}

\begin{bsp}
	Wir zeigen, dass $A \rightarrow B, B \rightarrow C \models A \rightarrow C$
	\marginpar{Sollte in Zeile 6 3,5 oder nur 5 stehen?}
	\begin{figure}[h]
	\centering
	
	
	\begin{tabular}{l r | l l l}
		$1$ & $A \rightarrow B$ &  $\Prae{1}$ & &\\
		$2$ & $B \rightarrow C$ &  $\Prae{2}$ & &\\
		$3$ & $A$ &  $\An{3}$ & &\\
		$4$ & $B$ &  $\Bsub$ & 1,3 & (1,3)\\
		$5$ & $C$ &  $\Bsub$ & 2,4 & (1,2,3)\\
		$6$ & $A \rightarrow C$ &  $\Esub$ & 3,5 & (1,2)\\
	\end{tabular}


	\end{figure}
\end{bsp}

\subsection{Disjunktion}

\begin{regel}{Disjunktionseinführung $(1)$}{}
	$\Eoder$: $\AA_L \implies (\AA \lor \BB)_L$		
\end{regel}

\begin{regel}{Disjunktionseinführung $(2)$}{}
	$\Eoder$: $\AA_L \implies (\BB \lor \AA)_L$		
\end{regel}


\begin{regel}{Disjunktionsbeseitigung}{}
	$\Boder$: $(\AA \lor \BB)_L, [\AA] \dots \CC, [\BB] \dots \CC \implies \CC$		
\end{regel}

\newpage

\begin{bsp} 
	Wir zeigen, dass $A \lor B, A \rightarrow C, B \rightarrow C \models C$
	\begin{figure}[H]
	\centering
	
	
	\begin{tabular}{l r | l l l}
		$1$ & $A \lor B$ &  $\Prae{1}$ & &\\
		$2$ & $A \rightarrow C$ &  $\Prae{2}$ & &\\
		$3$ & $B \rightarrow C$ &  $\Prae{3}$ & &\\
		$4$ & $A$ &  $\An{4}$ & &\\
		$5$ & $B$ &  $\An{5}$ & &\\
		$6$ & $C$ &  $\Bsub$ & $2,4$ & (2,4)\\
		$7$ & $C$ &  $\Bsub$ & $3,5$ & (3,5)\\
		$8$ & $C$ &  $\Boder$ & $1,6,7$ & (1,2,3)\\
	\end{tabular}


	\end{figure}
\end{bsp}



\end{document}
