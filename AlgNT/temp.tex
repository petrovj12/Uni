\documentclass[12pt, a4paper]{article}

\usepackage{amsthm,amsmath, amsfonts}
%\usepackage[ngerman]{babel}
\usepackage{tcolorbox}
\usepackage{hyperref, cleveref}
\usepackage{xcolor, thmtools}
\usepackage{enumitem}
%\usepackage{marginnote}
\usepackage{geometry}


\newcounter{aufgabe}

\geometry{
	left=1in,
	right=2.5in,
	top=2cm,
	bottom=3cm,
	marginparwidth=3cm,
	marginparsep=0.8cm,
	a4paper
}

%----Fonts----
\usepackage{libertinus}
\usepackage[libertine]{newtxmath}

%--Colors--
\definecolor{astral}{RGB}{3,57,108}
\definecolor{dblue}{RGB}{1,31,75}


%%%%%%%%%%%%%
%% General %%
%%%%%%%%%%%%%

\newcommand{\C}{\mathbb{C}}
\newcommand{\R}{\mathbb{R}}
\newcommand{\Q}{\mathbb{Q}}
\newcommand{\N}{\mathbb{N}}
\newcommand{\Z}{\mathbb{Z}}
\newcommand{\HH}{\mathbb{H}}
\newcommand{\ohne}{\setminus}
\newcommand{\card}[1]{\operatorname{\#}\left(#1 \right)}
\renewcommand{\iff}{\Leftrightarrow}
\renewcommand{\implies}{\Rightarrow}
\renewcommand{\tilde}{\widetilde}
\renewcommand{\bar}{\overline}
\newcommand{\aufz}[2]{#1_1, \dots, #1_{#2}}
\newcommand{\calp}{\mathcal{P}}
\newcommand{\vphi}{\varphi}
\newcommand{\norm}[1]{\lVert #1 \rVert}
\newcommand{\abs}[1]{\lvert #1 \rvert}
\DeclareMathOperator{\re}{Re}
\DeclareMathOperator{\imag}{Im}

%%%%%%%%%%%%%%%%%%%%%
%% Category Theory %%
%%%%%%%%%%%%%%%%%%%%%

\DeclareMathOperator{\Ob}{Ob}
\newcommand{\calc}{\mathcal{C}}
\DeclareMathOperator{\Set}{\mathbf{Set}}
\DeclareMathOperator{\CTop}{\mathbf{Top}}
\DeclareMathOperator{\CVec}{\mathbf{Vec}}
\DeclareMathOperator{\Grp}{\mathbf{Grp}}

%%%%%%%%%%%%%
%% Algebra %%
%%%%%%%%%%%%%

\DeclareMathOperator{\Grad}{Grad}
\DeclareMathOperator{\minpol}{minpol}
\DeclareMathOperator{\charpol}{charpol}
\DeclareMathOperator{\ord}{ord}
\DeclareMathOperator{\im}{im}
\DeclareMathOperator{\Bild}{Bild}
\DeclareMathOperator{\ggT}{ggT}
\DeclareMathOperator{\kgv}{kgV}
\DeclareMathOperator{\Gal}{Gal}
\DeclareMathOperator{\Aut}{Aut}
\DeclareMathOperator{\Bij}{Bij}
\DeclareMathOperator{\Abb}{Abb}
\DeclareMathOperator{\Mod}{Mod}
\DeclareMathOperator{\sgn}{sgn}
\DeclareMathOperator{\F}{\mathbb{F}}
\DeclareMathOperator{\id}{id}
\DeclareMathOperator{\Ker}{Ker}
\DeclareMathOperator{\Cl}{Cl}
\DeclareMathOperator{\Span}{Span}
\DeclareMathOperator{\spec}{Spec}
\DeclareMathOperator{\lok}{Lok}
\DeclareMathOperator{\OK}{\mathcal{O}_K}
\DeclareMathOperator{\OL}{\mathcal{O}_K}
\DeclareMathOperator{\einmat}{\mathbb{E}}
\DeclareMathOperator{\GL}{GL}
\DeclareMathOperator{\SL}{SL}
\DeclareMathOperator{\Orth}{O}
\DeclareMathOperator{\SO}{SO}
\DeclareMathOperator{\rank}{rank}
\newcommand{\pp}{\mathfrak{p}}
\renewcommand{\aa}{\mathfrak{a}}
\newcommand{\bb}{\mathfrak{b}}
\newcommand{\qq}{\mathfrak{q}}
\newcommand{\cc}{\mathfrak{c}}
\newcommand{\mm}{\mathfrak{m}}
\newcommand{\zpz}{\Z / p\Z}
\newcommand{\zmz}{\Z / m\Z}
\newcommand{\znz}{\Z / n\Z}
\newcommand{\zNz}{\Z / N\Z}
\newcommand{\calt}{\mathcal{T}}
\newcommand{\caln}{\mathcal{N}}
\newcommand{\Zi}{\Z[i]}
\newcommand{\Qi}{\Q[i]}
\newcommand{\zmod}[1]{\Z / #1\Z}
\newcommand{\kxn}[1]{k[x_1, \dots, x_{#1}]}
\newcommand{\kx}{k[x]}
\newcommand{\kyn}[1]{k[y_1, \dots, y_{#1}]}
\newcommand{\ky}{k[y]}
\newcommand{\gen}[1]{\langle #1 \rangle}
\newcommand{\epim}{\twoheadrightarrow}
\newcommand{\monom}{\hookrightarrow}
\newcommand{\nsub}{\vartriangleright}
\newcommand{\nsubl}{\vartriangleleft}

%%%%%%%%%%%%%%
%% Analysis %%
%%%%%%%%%%%%%%

\newcommand{\eps}{\varepsilon}
\newcommand{\veps}{\epsilon}
\DeclareMathOperator{\holo}{\mathcal{O}}
\DeclareMathOperator{\vol}{vol}

%%%%%%%%%%%%%%
%% Theorems %%
%%%%%%%%%%%%%%

%% Boxed Theorems %%
\tcbuselibrary{theorems}
\newtcbtheorem[number within=section, crefname={\mathrm{Satz}}{Satz}]{satz}{Satz}
{before skip=2em, after skip=2em, colback=red!5!white, colframe=red!50!black}{S}
\newtcbtheorem[number within=section, crefname={\mathrm{Theorem}}{Theorem}]{thm}{Theorem}
{before skip=2em, after skip=2em, colback=red!5!white, colframe=red!50!black}{S}
\newtcbtheorem[number within=section, crefname={\mathrm{Korollar}}{Korollar}]{kor}
{Korollar}{before skip=2em, after skip=2em}{K}
\newtcbtheorem[number within=section, crefname={\mathrm{Corollary}}{Corollary}]{cor}
{Corollary}{before skip=2em, after skip=2em}{C}
\newtcbtheorem[number within=section, crefname={\mathrm{Definition}}{Definition}]{Def}
{Definition}{before skip=2em, after skip=2em, colback=green!5!white, colframe=green!50!black}{D}
\newtcbtheorem[number within=section, crefname={\mathrm{Hilfssatz}}{Hilfssatz}]{hilfs}
{Hilfssatz}{before skip=2em, after skip=2em}{HS}
\newtcbtheorem[number within=section, crefname={\mathrm{Lemma}}{Lemma}]{Lemma}{Lemma}
{before skip=2em, after skip=2em}{L}
\newtcbtheorem[number within=section, crefname={\mathrm{Folgerung}}{Folgerung}]{folg}
{Folgerung}{before skip=2em, after skip=2em}{FOLG}
\newtcbtheorem[number within=section, crefname={\mathrm{Aufgabe}}{Aufgabe}]{auf}{Aufgabe}{before skip=2em, after skip=2em, colback=blue!5!white, colframe=blue!50!black}{A}
\newtcbtheorem[number within=section, crefname={\mathrm{Proposition}}{Proposition}]
{prop}{Proposition}{before skip=2em, after skip=2em, colback=yellow!5!white, colframe=brown!50!black}{P}

%% Normal Theorems %%
\declaretheoremstyle[
	headfont=\normalfont,
	spaceabove=1em,
	spacebelow=1em
]{beisp}
\declaretheoremstyle[
	headfont=\color{purple}\normalfont,
	spaceabove=1em,
	spacebelow=1em
]{beme}
\declaretheoremstyle[
	headfont=\color{astral}\normalfont\bfseries,
	spaceabove=1em,
	spacebelow=1em
]{aufg}
\declaretheorem[
	style=beisp,
	name=Beispiel,
	numberwithin=section
]{bsp}
\declaretheorem[
	style=beisp,
	name=Example,
	numberwithin=section
]{ex}
\declaretheorem[
	style=beisp,
	name=Remark,
	numberwithin=section
]{rem}
\declaretheorem[
	style=beme,
	name=Bemerkung,
	numberwithin=section
]{bem}
\declaretheorem[
	style=aufg,
	name=Aufgabe,
]{Aufgabe}




\usepackage{tikz-cd}

\renewcommand{\phi}{\varphi}
\newcommand{\BB}{\mathscr{B}}


\DeclareMathOperator{\Frac}{Frac}
\DeclareMathOperator{\tr}{tr}
\DeclareMathOperator{\Tr}{Tr}
\DeclareMathOperator{\Nm}{Nm}
\DeclareMathOperator{\FracId}{Frac\;Id}
\DeclareMathOperator{\disc}{disc}
\DeclareMathOperator{\covol}{covol}
\DeclareMathOperator{\Rgez}{\R_{\geq 0}}
\DeclareMathOperator{\Norm}{N}

\begin{document}

\title{Algebraic Number Theory}
\author{Lecture Notes}
\maketitle


\tableofcontents

\section{Integral elements}

\begin{Def}{}{}
	$\phi: A \to B$, $b \in B$ is integral over $A$ iff $\exists f \in A[t]$ 
	monic with $f(b) = 0$. The ring $B$ is integral over $A$ 
	if all $b \in B$ are integral over $A$.
\end{Def}

\begin{ex}
	$\sqrt{2} \in \Q[\sqrt{2}]$ is integral over $\Z$, $\frac{1}{2} \in \Q$ 
	is not integral over $\Z$.
\end{ex}

\begin{prop}{}{}
	$\phi: A \to B$ then the following are equivalent:

	\begin{enumerate}[label = (\roman*)]
		\item $b$ is integral over $A$
		\item $A[b]$ is finitely generated as an $A$-module.
		\item $A[b] \subset C \subset B$, $C$ is finitely generated as an 
			$A$-module.
		\item There exists a faithful $A[b]$-module $M$ finite 
			as an $A$-module.
	\end{enumerate}
\end{prop}

\begin{Def}{}{}
	$A \subset B$, $\overline{A} = \{b \in B \mid b \text{ integral over } A\} $
	is called the integral closure of $A$ in $B$.
\end{Def}

\begin{cor}{}{}
	$\overline{A}$ is a ring.
\end{cor}

\begin{prop} {}{}
 $A \subset B$, $B \subset C$ are integral $\implies A \subset C$ is integral.
\end{prop}

\begin{cor}{}{}
	$A \subset B$ then $\overline{\overline{A}} = A$.
\end{cor}

\textsc{Our objects of study: } $\OK$. It is clearly integrally closed.

\begin{figure}[h]
\centering


\begin{tikzcd}
	\Q\arrow[r] & K \\
	\Z\arrow[u]\arrow[r] & \overline{\Z} \arrow[u] =: \OK
\end{tikzcd}
\caption{$\OK$}
\end{figure}

Remark: $A$ is UFD then $A = \overline{A} \subset \Frac(A)$

\begin{proof}
	Same as for $\Z \subset \Q$.
\end{proof}

\begin{prop}{}{}
	$A = \overline{A} \subset K = \Frac(A)$. $L \vert K$ separable field 
	extension. Then $l \in L$ integral over $A$ $\iff f_l \in A[t]$. 
\end{prop}

\begin{ex}
	$K = \Q[\sqrt{5}]$  $\OK = ?$. Let $x = a + b\sqrt{5} \in \OK \subset K$. 
	Then $f_x = (X-x)(X-\overline{x}) = X^2 - 2aX + a^2 - 5b^2$. 
	Thus $2a, a^2 - 5b^2 \in \Z$. From this one can calculate that 
	$\OK = \Q[\frac{1 + \sqrt{5}}{2}] \neq \Q[\sqrt{5}]$.
\end{ex}

\section{Free A-modules}

\begin{thm}{Structure theorem for finitely generated Abelian groups}{}
	Any finitely generated $\Z$-module $M$ is isomorphic to $M = \Z^r \oplus 
	\Z/(d_1) \oplus \dots \oplus$, $d_i \mid d_{i+1}$. In other terms 
	$M = \Z^r \oplus \bigoplus_i \Z/(p_i^{e_i})$
\end{thm}

\noindent Construction: $B$ is an $A$-algebra free as an $A$-module. 
Let $b \in B$. 
$\tr(b) := \tr(M_b)$, $\Nm(b):=\det(M_b)$ where $M_b$ is the matrix of the multiplication 
map $\cdot b$. Then $\tr: B \to A$ is additive and $A$-linear and 
$\Nm: B \to A$ is multiplicative.

\begin{prop}{}{}
	$L \vert K$ finite extension, $[L:K] = n$. Let $l \in L$ 
	with minimal polynomial $f_l(x) = x^m - 
	a_1x^{m-1}+ \dots \pm a_{m}$. Let $s = \frac{n}{m}$. Then 
	$\tr(l) = sa_1$, $\Nm(l) = a_m^s$.
\end{prop}

\section{Bilinear forms}

Let $M$ be free as an $A$-module and let $\Psi: M \times M \to A$
be a bilinear form with Gram-matrix $G = (g_{ij})$, i.e. 
$g_{ij} = \Psi(e_i, e_j)$. The discriminat of $\Psi$ with respect 
to the standard basis is $\disc \Psi = \det(G)$.
Let $B$ be a change-of-basis matrix. Then the Gram-matrix 
with respect to the new basis is given by $G' = B^t G B$ meaning that
$\det(G) = \det(B)^2 \det(G)$. This means that usually for rings the 
determinant of the Gram-matrix is not independent of the chosen basis. 
However if $A = \Z$, then the discriminant is independent of bases!

For a field extension $L \vert K$ we define $\disc_{L\mid K} := \disc \Tr$, 
where $\Tr$ is the trace form $\Tr: L \times L \to K$, $(l, l') 
\mapsto \tr(l \cdot l')$.

\begin{ex}
	$L = \Q[\sqrt{d}]$, $\mathcal{B} = (1, \sqrt{d})$, $\Tr = 
	\begin{pmatrix}
		\tr(1) & \tr(\sqrt{d}) \\
		\tr(\sqrt{d}) & \tr(\sqrt{d})
	\end{pmatrix}$. Then $\disc_{L \vert K}(\mathcal{B}) = 4d$.
\end{ex}

\begin{prop}{}{}
	$L \vert K$ finite separable extension of degree $n = [L:K]$. Let 
	$\sigma_i: L \to \overline{L}$ be the embeddings of $L$ in an 
	algebraically closed field (or even the normal closure of $L$).
	Then for any $l \in L$, $tr(l) = \sum_{i=1}^{n} \sigma_i{l}$ and 
	$\Nm(l) = \prod_{i=1}^{n}\sigma_i(l)$.
\end{prop}

In particular, if $L \vert K$ is Galois, then $\tr(l) = \sum_{\sigma \in 
\Gal(L\vert K)} \sigma(l)$. 

$\Psi: M \times M \to A$, $\disc(\Psi) = 0 \iff:$ $\Psi$ is degenerate
	$\iff \Psi(m, \cdot) \equiv 0$ for some $m \neq 0$ $\iff$ $\Psi(\cdot, m) 
	\equiv 0$ for some $m \neq 0$. We need the trace form to be 
	non-degenerate for separable $L \vert K$.


	\begin{thm}{Dedekind's theorem on the independence of characters}{}
		Let $K$ be a field, $G$ a group and $\chi_i: G \to K*$ pairwise 
		different. Then $\{\chi_i\}_{i \in I} $ is linearly independent over $K$.
	\end{thm}

	\begin{prop}{}{}
		Let $A = \overline{A} \subset K$ and $B$ be the integral closure of 
		$A$ in $L$ for a finite field extension $L\vert K$. Then for 
		any $b \in B$, $\tr_{L \vert K}(b), \Nm_{L \vert K}(b) \in A$. 
	\end{prop}

	\begin{proof}
		Since $b \in B$, $\minpol(b) \in A[X]$.
	\end{proof}

	\begin{prop}{}{}
		Let $L\vert K$ be a finite separable field extension, $n = [L:K]$. 
		Let $\sigma_i$ be the embeddings of $L$ in its Galois closure. 
		For a basis $\mathcal{B} = (l_1, \dots, l_n)$ of $L$ as a 
		$K$-vector space we have $\disc_{L \vert K}^{\mathcal{B}} = 
		\det^2((\sigma_i l_j)_{ij}) \neq 0$.
	\end{prop}

	\begin{proof}
		$\det(\Tr_{L\vert K}) = \det((\tr(l_il_j)_{ij})) \det((\sum_{k}
	\sigma_k(l_il_j))_{ij}) = \det((\sum_{k}\sigma_k$ $(l_i) \sigma_k(l_j))) = 
	\det^2(\sigma_k(l_i)) $. Let $M = (\sigma_kl_j)_{kj}$
	Suppose $\det M = 0$. Then the rows are linearly dependent, meaning
	$\sum \lambda_i \sigma_i(l_j) = 0 \forall  j$. Thus $\sum \lambda_i \sigma_i$
	$\equiv 0$, contradicting Dedekind's theorem on the independence of 
	characters.
	\end{proof}

	\begin{thm}{}{}
		Let $A \subset K$ be integrally closed, $L \vert K$ finite and separable
		of degree $n$. Then the integral closure $B$ of $A$ in $L$ is a 
		finitely generated $A$-module of rank $n$. The rank of a module $B$
		is defined as $\dim_K(K \otimes_A B)$. Furthermore
		\begin{itemize}
			\item if $A$ is Noetherian, so is $B$.
			\item if $A$ is a PID then $B \cong A^{\oplus n}$
		\end{itemize}
	\end{thm}

	\begin{cor}{}{}
		If $A = \Z$, $K \vert \Q$ a finite field extension, then 
		$\OK = \Z^n$ where $n = [K : \Q]$.
	\end{cor}


	\begin{Def}{}{}
		A basis of $\OK$ as a $\Z$-module  is called an integral basis 
		of $\OK$.
	\end{Def}

	\begin{ex} \
		\begin{itemize}
			\item Let $K = \Q[\sqrt{d}]$ and $\mathcal{B} = (1, \sqrt{d})$ 
				\marginpar{not necessarily an integral basis!} 
				which is a basis of $K$ consisting of integral elements.
				Then $\disc_{K\vert Q}^{\mathcal{B}} = \det \begin{pmatrix}
					\tr(1) & \tr(\sqrt{d}) \\
					\tr(\sqrt{d})& \tr(d))
				\end{pmatrix} = 4d$. Let $\mathcal{B}' = (e_1, e_2)$ be an integral 
				basis of $\OK$. Then the elements of $\mathcal{B}$ can be written as a 
				linear combinations over $\Z$ of the elements of $\mathcal{B'}$ 
				meaning $\mathcal{B} = M \mathcal{B}'$ for some matrix $M$. Thus 
				$4d = \disc^{\mathcal{B}} = \det^2(M) \disc^{\mathcal{B}'}$. 
				Thus $\det(M) \mid 2$. If $|\det(M)| = 2$, $\Z(\mathcal{B}) 
				\monom \Z(\mathcal{B}')$ is of index $2$.  Thus 
				the candidates to be checked for an integral basis 
				of $\OK$ are $\frac{1}{2}, \frac{\sqrt{d}}{2}, \frac{1+\sqrt{d}{}}{2}$
				. Since the first two can be easily discarded, we get 
				$(1, \frac{1+\sqrt{d}}{2})$ as an integral basis.
			\item Let $\mathcal{B} = (e_1, \dots, e_n)$ be a 
				basis of $K\vert \Q$ consisting of integral elements.
				$\disc_{L \vert K}^{\mathcal{B}} = \pm \prod_{i=1}^{n} p_i{e_i}$. 
				If $e_i < 2$, then $\mathcal{B}$ is automatically 
				an integral basis. Else if there is only one 
				prime $p := p_i$ with $e_i = 2$ and the rest $e_j < 2$, 
				we only have to check elements of the form 
				$\sum_{i=1}^{n}\frac{n_il_i}{p}$ to find an integral basis.
		\end{itemize}
	\end{ex}

	Let $f \in K[X]$ with $f = \prod (X \alpha_i)$ in 
	$\overline{K}$. $\Delta(f) := \prod_{i < j} 
	(\alpha_i - \alpha_j))$ which is a polynomial 
	in the coefficients of $f$. If 
	$\deg f = 2$ and $f\in \R[X]$. Then if $x_1 = z
	\in \C \ohne \R$.
	then $x_2 = \overline{z}$, meaning $z -\overline{z}
	\in i\R \implies (z-\overline{z})^2 \in \R_{\leq 0
	}$.
If $\deg f = 3$ and $\alpha_1$ is real and $\alpha_2$
is not real, then $\alpha_3 = \overline{\alpha_1}$.
Thus $\Delta(f) \in i\R$.
\begin{prop}{}{}
	$\OK$ is the maximal subring of $K$ finitely 
	generated as a $\Z$-module.
\end{prop}

\begin{proof}
	Let $B \subset K$ be a finitely generated 
	$\Z$-module. Then by a previous theorem 
	since $\Z$ is a PID we have $B = \Z^n$.
	Let $b \in B$. Thus $\Z[b]$ is free and finite as a
	$\Z$-module meaning that
	$b$ is integral over $\Z$.
	Thus $b \in \OK$.
\end{proof}

\begin{prop}{}{}
	Let $L\vert K$ be a field extension. 
	Let $L = K[x]$ and $f_x$ be the minimal 
	polynomial of $x$ over $K$. With $\deg f = n$
	and $\BB = (1,x, \dots, x^{n-1})$ basis of 
	$L$ let $x_i = \sigma_i(x)$ be the different 
	images of $x$ under embeddings of $L$ in 
	its algebraic closure. Then 
	$$\disc_{L \vert K }^\BB = \prod_{i < j} 
	(x_i-x_j)^2 = (-1)^{n(n-1)/2} \Nm(f'(x))$$
\end{prop}

\begin{ex}{}{}
		Let $f = X^n - a$ and $L = K[x] \vert K$
		with $x^n=a$, $n \neq = 0 \in L$.
		Then $g = f' = n x^{n-1} =  n\frac{x^n}{x} 
		= \frac{na}{x}$. Clearly then $y \in 
		K[x]$. Since $x = \frac{na}{y} \in K[y]$
		we have $K[x] = K[y]$.
		We need now only the minimal polynomial 
		of $y$. $0 = f(x) = f(\frac{na}{y}) = 
		(\frac{na}{y})^n - a = 0 \implies y^n - n^n a^{n-1} = 0$.
		Thus $\Nm(y) = \pm n^n \cdot a^{n-1}$ and $\disc^\BB = \pm n^n a^{n-1}$. 
\end{ex}

\begin{cor}{}{}
	If $f = X^n + aX + b$ then $\disc(f) = \_\ a^n + \_\ b^{n-1}$.
\end{cor}

	By a simple calculation \marginpar{meaning calculation to be inserted} 
	$$\boxed{\disc(f) = (-1)^{{n \choose 2}}(1-n)^{n-1} a^n + (-1)^{n \choose 
	2}n^nb^{n-1}}$$

	We thus get 

	\begin{align*}
		\disc(X^2 + aX + b) &= a^2-4b \\
		\disc(X^3 + aX + b) &= -27b^2-4a^3 \\
		\disc(X^5 + aX + b) &= 5^5b^4-4^4a^5
	\end{align*}

	\begin{ex}
		$\disc(X^5-X-1) = 19 \cdot 5 \cdot 3$ meaning that $\BB = (1, \cdot, x^4)$
		 is an integral basis of $\OK$.
	\end{ex}

	\begin{prop}{}{}
		Let $K\vert \Q$ be a finite field extension and $\BB$ a 
		$\Q$-basis of $K$. Then $\sgn(\disc^\BB) = (-1)^s$ where 
		$s$ is the number of non-real embedding of $K$ in $\C$.
	\end{prop}

	\begin{thm}{Stickelberger's theorem}{}
		Let $K \vert \Q$ be a finite extension of degree $n$ and $\BB$ be a 
		basis of integral elements. Then $\disc^\BB \equiv 0, 1 \pmod 4$.
	\end{thm}

	\section{Dedekind rings}

	\begin{ex}{}{}
		Let $K \vert \Q$ be a finite extension of degree $n$.
		Then 

		\begin{itemize}
			\item $\OK = \Z^n$
			\item is Noetherian
			\item is integrally closed
			\item $\Frac(\OK) = K$.
		\end{itemize}

		Let $\aa \subset \OK$ be a non-zero ideal. Pick a non-zero element 
		$a \in \aa$. Then $\Z^n = (a) \subset \aa \subset \OK = \Z^n$ 
		thus $\aa = \Z^n$ is a free $\Z$-module. This implies that 
		$\OK/\aa$ is finite. In fact if we take a prime ideal $0 \neq \pp 
		\subset \OK$ then $\pp$ is maximal since every finite domain is a field.
	\end{ex}

	\begin{rem}
		This means that $\OK$ is of (Krull-)dimension 1.	
	\end{rem}

	\begin{Def}{}{}
		A ring $A$ is a Dedekind ring if $A$ is an integrally closed, Noetherian 
		ring of dimension one.
	\end{Def}

	\begin{ex}
		$\OK, K[x]$ are Dedekind.
	\end{ex}

	\begin{Lemma}{}{}
		Let $A \subset B$ be a ring extension such that $B$ is a finite 
		$A$-module. Then $A$ is a field $\iff$ $B$ is a field.
	\end{Lemma}

	\begin{prop}{}{}
		Let $A \subset K := \Frac(A)$ be a Dedekind ring and $L\vert K$ be a 
		finite separable field extension. Then the integral closue $
		B=\overline{A}\subset L$ is also a Dedekind ring. 
	\end{prop}

	\marginpar{\color{red} Section on local properties missing}

	\section{Valuations and DVR's}

	\begin{Def}{}{}
		A valuation $v$ on a field $K$ is a group homomorphism $v: K^{*} \to G$
		where $G$ is a totally ordered Abelian group satisfying $v(\lambda 
		+ \mu) \geq \min(v(\lambda), v(\mu ))$. A valuation is discrete 
		is $G = \Z$. 
	\end{Def}

	\begin{rem}
		We sometimes assume that $v: K^{*} \to \Z$ is surjective.
	\end{rem}

	\begin{Def}{}{}
		A domain $A$ is a discrete valuation ring (DVR) if $A = v^{-1}(\N_{\geq 0}
		)$ 
		for some discrete valuation on $\Frac(A)$.
	\end{Def}
	\begin{ex}
		Let $K = \Q$ and fix a prime number $p$. Then for $\frac{a}{b} = 
		p^m \frac{\overline{a}}{\overline{b}}$ with $(\overline{a}, \overline{b}) 
		= (\overline{a}, p) = (\overline{b}, p) = 1$ we assign $v_p(\frac{a}{b}) 
		= m$. This is a discrete valuation. Its DVR is $v_p^{-1}(\N) = 
		\{\frac{a}{b} \mid p \nmid b\} = \Z_{(p)} $.
	\end{ex}

	\begin{prop}{}{}
		Let $v: K^* \to \Z$ be a valuation and $A = v^{-1}(\N)$. Let 
		$t \in v^{-1}(1)$.

		\begin{enumerate}[label = (\roman*)]
			\item $A$ is noetherian
			\item $m = t$ is the only maximal ideal
			\item $\aa \in A$ ideal then $\aa = (t^n)$.
			\item $A$ is integrally closed
		\end{enumerate}
	\end{prop}

	\begin{thm}{}{}
		Let $(A, \mm)$ be a local Dedekind domain. Then $A$ is a DVR.	
	\end{thm}

	\begin{rem}
		Let $A$ be a Dedekind ring and $\mm \subset A$ maximal. Then 
		$A_\mm$ is a local Dedekind ring $\implies A_\mm$ is a DVR.
	\end{rem}

	\begin{thm}{}{}
		Let $A \subset K$ be a Dedekind ring, and $\aa \subset A$ be an ideal.	
		Then $\aa = \prod_{i=1}^{n} \mm_i^{e_i}$. This product is unique upto refactoring.
	\end{thm}

	\begin{Lemma}{}{}
		Let $A$ be a Dedekind ring, then any ideal $\aa \subset A$ contains a product 
		of maximal ideals.
	\end{Lemma}

	\begin{Lemma}{}{}
		Let $A$ be a ring $\mm \subset A$ maximal. Let $\mm_\mm$ be the maximal ideal 
		of the local ring $A_\mm = (A \ohne \mm)^{-1}A$.
		Then for any $n \in \N$ 
		$A/\mm^n \cong A_\mm/\mm_\mm^n$.
	\end{Lemma}

	\section{Projective modules over Dedekind rings}

	\begin{Def}{}{}
		$M \in \Mod(A)$	is \emph{projective} if for every modules $N$ and $N^*$ 
		with a surjection $\pi: N \epim N^*$ and homomorphism $\alpha: M \to N^*$ there 
		exists a so called \emph{lift} $\tilde{\alpha}: M \to N$ such that 
		$\alpha = \pi \circ \tilde\alpha$.
	\end{Def}

	\begin{ex} \
		\begin{itemize}
			\item $M = A^n$ is projective.
			\item $M$ is projective $\implies$ $M \oplus M'$ is free for some $M'$.
		\end{itemize}	
	\end{ex}

	\begin{cor}{}{}
		$0 \neq \aa \subset A$ ideal is a Dedekind ring. Then $\aa$ is projective 
		as an $A$-module.
	\end{cor}

	\begin{cor}{Existence of inverse}{}
		For any $\aa \neq 0$ there exists an ideal $\bb$ such that $\aa \bb = (c)$ for 
		some $c \in A$.
	\end{cor} 

	\section{Ideal class group}

	\begin{Def}{}{}
		A fractional ideal is finitely generated $A$-module in $K = \Frac A$.	
	\end{Def}

	\begin{rem}{}{}
		$M = \langle \frac{a_i}{s_i}\rangle$. Let $s = \prod_{i=1}^{N} a_i$. 
		Then $sM = \langle a_i \rangle = \aa$ is an ideal. Thus 
		all fractional ideals are of the from $\frac{1}{s} \aa$ for some ideal $\aa$.
	\end{rem}

	\begin{rem}{}{}
		Let $\lambda, \mu \in K^*$.	Define $(\lambda) = A\lambda$. 
		Then $(\lambda) = (\mu) \iff \mu \in (\lambda) \land \lambda \in (\mu)
		\iff \lambda = u \mu$ for some unit $u$.
	\end{rem}

	\begin{Def}{Ideal class group}{}
		Define the \emph{ideal class group} $\Cl_K$ through the following 
		exact sequence $0 \to A^* \to K^* \to \FracId(A) \to \Cl_K \to 0$, 
		where $\FracId$ is the set of all fractional ideals. Then $\Cl_K$ is a group.
	\end{Def}


	\begin{Def}{}{}
		Let $A \subset K = \Frac A$ be a Dedekind ring and $L \vert K$ be 
		a finite field extension. Let $B$ be the integral closure of $A$ in 
		$L$. Then $B$ is Dedekind. Let $\pp$ be a prime ideal of $A$. 
		Then $\pp^e = \prod_{i=1}^{g} \pp_i^{e_i}$. The $e_i$ are called 
		the \emph{ramification index} of $\pp_i$. $f_i = [B/\pp_i : A/\pp]$ 
		is called the \emph{inertial degree}.
	\end{Def}

	\begin{thm}{}{}
		$n = \sum_{i=1}^{g} e_if_i$. If $L \vert K$ is Galois then 
		the Galois group $G$ acts transitively on the prime ideals 
		$\pp_i$. We then have $e_i = e$, $f_i = f$ for all $i$ and 
		thus $n =efg$.
	\end{thm}

	\begin{prop}{}{}
		Let $A \subset K$ be Dedekind.
		\begin{itemize}
			\item $\aa \subset A$ is projective
			\item $M$ is projective $\implies M \cong A^{n-1} \oplus \aa$.
			\item $A^{n-1}\oplus \aa \cong A^{n-1} \oplus \bb \iff \aa \sim \bb$.
			\item $\forall S \subset A$ $S^{-1}A \otimes_A M \cong S^{-1}M$.
		\end{itemize}
	\end{prop}

	\begin{thm}
		Let $A \subset K$ be Dedekind, $L \vert K$ finite field extension and 
		$B$ the integral closure of $A$ in $L$. Assume that $B \cong A^n$ 
		and that $\forall \pp \subset A$ $A/\pp$ is perfect. Let 
		$\BB$ be a basis of $B$ as an $A$-module.
		Then $(\Delta_{L \vert K}) = \prod_{i=1}^{n} \pp_i^{e_i}$. Then 
		$\pp \subset A$ is ramified $\iff$ $\pp$ is one of the $\pp_i$.
	\end{thm}

	\begin{cor}{}{}
		Let $K \vert \Q$ be finite. Then there are only finitely many primes
		dividing $\Delta_{K \vert \Q}$, meaning that only finitely many 
		primes ramify.
	\end{cor}

	\section{Norm of an ideal}

	\begin{Def}{}{}
		Let $L \vert K$ a finite field extension and $\qq \subset \OL$. 
		Then $\pp = \OK \cap \qq$ is also prime. Let $[\OL/\qq:\OK/\pp] = f$. 
		Then the define the \emph{norm} of $\qq$ by $\Nm(\qq) = \qq^f$.
		Define $\Nm(\prod_{i} \qq_i^{e_i}) = \prod_{i} \Nm(\qq_i)^{e_i}$.
	\end{Def}
	\begin{prop}{}{}
		\begin{itemize}
			\item $\aa \subset \OK$ then $\Nm(\aa^e) = \aa^n$.
			\item $L \vert K$ Galois, then $\Nm(\aa) = \prod_{\sigma \in G} \sigma(\qq)$.
			\item $\Nm_{M \vert K} = \Nm_{L \vert K} \circ \Nm_{M \vert L}$ for a 
				tower $K \to L \to M$ of fields.
			\item $b \in \OL$ then $\Nm((b)) = (\Nm(b))$.
		\end{itemize}
	\end{prop}

	\section{Latices}	

	\begin{Def}{Latice}{}
		Let $V \cong \R^n$ be a Euclidean vector space $(V, \langle \cdot \rangle)$.
		A subgroup $\Lambda \subset V$ is called a \emph{latice} if 
		\begin{itemize}
			\item $\Lambda \cong \Z^n$
			\item $\Lambda$ is spanned by a basis of $V$
		\end{itemize}
	\end{Def}

	\begin{ex}
		$\Z^n \subset \R^n$. If $\BB_lambda = (l_1, \dots, l_n)$ is a latice basis, 
		then for any other basis $\BB_\Lambda' = A \cdot \BB_\Lambda$ with 
		$A \in \GL_n(\Z)$ 
	\end{ex}

	\begin{Def}{}{}
		Let $\Lambda$ ba lattice. The \emph{covolume} of $\Lambda$ for a basis 
		$\BB_\Lambda = (l_i)$ is the determinant $\det(G)$ where 
		$G = (g_{ij})$ with $g_{ij} = \langle l_i, l_j \rangle$.
	\end{Def}

	\begin{rem}
		\begin{itemize}
		\item For a different basis $\BB' = A\BB$ we have $\deg(G') = \det(A^t G A) =
		\det(A)^2 \det(B)$. 
		\item $\covol(\Lambda) = \vol(V/\Lambda) = \vol(F)$ where $F = 
			\{\sum_{i=1}^{n} x_il_i
		\mid x_i \in [0,1]\} $.
		\end{itemize}
	\end{rem}

	\begin{Def}{}{}
		A subset $K \subset V$ is called 
		\begin{itemize}
			\item \emph{central} if $x \in K \iff -x \in K$.
			\item \emph{convex} if $x,y \in K \implies tx + (1-t)y \in K \forall t \in [0,1]$.
		\end{itemize}
	\end{Def}

	\begin{thm}{Minkowskis latice point theorem}{}
		Let $K \subset V$ be a compact, convex and central subset such that 
		$\mu(K) \geq 2^n\mu(F_\Lambda)$. Then $K$ contains a non-zero latice point of 
		$\Lambda$.
	\end{thm}

	Let $K \vert \Q$ be a number field of degree $n$. Since the extension 
	is separable there exists an $\alpha \in K$ with $K = \Q(\alpha)$ 
	with $f = f_\alpha$ its minimal polynomial.
	We will construct $V_K \cong \R^n$. For any root $\alpha_i$ of $f$ 
	we get an embedding $K \to \C$. Let $r$ be the number of 
	purely real embeddings and $s$ be the number of pairs of non-real 
	complex embeddings. Then $n = r + 2s$. Let now 
	$$V_K = \bigoplus_{\sigma_i: K \to \R} \R \bigoplus_{\sigma_i:K \to \C} \C$$
	Then $\dim_\R V_k = r + 2s = n$. Consider $v \in V_K$ with $v = (x_1, \dots, x_r, 
	z_1, \dots, z_s)$. Define 
	$$\norm{v}^2 = \sum_{i=1}^{k} x_i^2 + 2\sum_{i=1}^{s}\norm{z_i}^2$$.
	Let $\nu:K\to V_K$ be given by $\nu(a)_i = \sigma_i(a)$.
	\begin{thm}{}{}
		$\nu(\OK)$ is a lattice in $V_K$ of covolume $2^{-s}\sqrt{\abs{\Delta_K}}$.
	\end{thm}

	\begin{thm}{}{}
		Let $K \vert \Q$ be finite. Then for any class $[\aa] \in \Cl(\OK)$
		there exists an ideal $\aa \in [\aa]$ such that
		\begin{itemize}
			\item $\aa \subset \OK$.
			\item $\Nm(\aa) \leq \frac{n!}{n^n} (\frac{4}{\pi})^s \sqrt{\abs{\Delta_K}}$
		\end{itemize}
	\end{thm}

	\begin{cor}{}{}
		$\Cl(\OK)$ is finite.
	\end{cor}

	\begin{cor}{}{}
		Let $K \vert \Q$ be a number field of degree $n \geq 2$. Then 
		$\Delta_K \neq 1$.
	\end{cor}

	\begin{prop}{}{}
		\begin{itemize}
			\item $\covol(\sigma(\Nm(\aa))) \sqrt{\abs{\Delta_K}}$.
			\item $x_i \in \R_{\geq 0}$
		\end{itemize}
	\end{prop}

	\begin{prop}{AM-GM Inequality}{}
		For $x_i \in \Rgez$ we have 
		$\sqrt[n]{\prod_{i} x_i} \leq \frac{\sum_{i} x_i}{n}$	
	\end{prop}

	\begin{proof}
		WLOG we may assume that $\sum_{i} x_i = 1$. 
		Let $K := \{x \in \Rgez \mid \sum_{i} x_i = 1\} $.
		The set is clearly compact. Consider now $f: K \to \R$
		given by $(x_i) \mapsto \prod_{i} x_i$. Since the function 
		is continuous, there must a maximum. Then $\nabla f 
		= (\prod_{i} x_i) (\frac{1}{x_1}, \dots, \frac{1}{x_n})^t$.
		Let $x$ be the maximum of $f$ and $t \in \R^n$ arbitrary.
		$f(x + \eps t) = f(x) + \eps t \cdot \nabla F$ for $\eps \to 0$.
		Thus $t \perp \nabla f(x)$. Choosing $t = (1, -1, 0, \dots, 0)^t$
		we get $x_1 = x_2$. Similarly $x_1 = \dots = x_n = \frac{1}{n}$.
		Thus we get $\prod_{i} x_i \leq \frac{1}{n}$.
	\end{proof}

	$K = K(t) := \{x \in V_K \mid \sum_{i=1}^{r} \abs{x_i} + 
	2\sum_{i=1}^{s} \norm{x_{r+1}}\leq t\} $. Then $K$ is closed.
	Furtehrmore $\vol(K(t)) = t^n \vol(K(1))$.

	\begin{prop}{}{}
		$\vol K(1) = \frac{2^r(\frac{\pi}{2})^2}{n!}$.
	\end{prop}

	\begin{thm}{}{}
		For anny class $\eta \in \Cl(\OK)$ there exists a non-fractional ideal
		$\aa \in \eta$ with $\Nm(\aa) \leq \frac{n!}{n^n}(\frac{4}{\pi})^s
		\sqrt{\abs{\Delta_K}}$. 
	\end{thm}

	\begin{thm}{}{}
		A subgroup $\Gamma \subset V$ is discrete $\iff$ $\Gamma$ is a lattice.
	\end{thm}

	\section{Unit theorem}
	
	Main result to be proved:
	\begin{thm}{Unit theorem}{}
		Let $K \vert \Q$ be finite of degree $n = r + 2s$. Then $u_K = \OK^* 
		= \mu(K) \oplus \Z^{r+s-1}$ where $\mu(K) = \{\lambda \in K 
		\mid \exists n(\lambda) : \lambda^{n(\lambda)} = 1\} $.
	\end{thm}

	\begin{ex}
		$\mu(\Q(i)) = \gen{i}$.
	\end{ex}

	\begin{Lemma}{Easy finiteness lemma}{}
		Let $M \in \R$ and $m \in \N$. Then 
		$$\card {\{z \in \C \mid \text{ integral over } \Z; \deg(z) \leq m 
		; \norm{\sigma_i z} \leq M \forall  i\}} < \infty$$
	\end{Lemma}

	\begin{prop}{}{}
		LEt $z \in K^*$. Then we have 
		$z \in \mu(K) \iff \norm{\sigma(z)} = 1 \forall \sigma: K \to \C$
	\end{prop}

	\begin{rem}
		$\mu(K) \subset U_K = \OK^* \subset \OK$ 
	\end{rem}

	\begin{thm}{}{}
		$\rank U_K = r+s-1$. In particular $U_K \cong \mu(K) \oplus \Z^{r+s-1}$.
	\end{thm}
	
	\section{Regulator}

	Let $K \vert \Q$ be a number field. Then $\OK^* = \Z/(m) \oplus \Z^{r+s-1}$. 
	We can embed $L: \OK \to \R^{r+s}$ by $L(x) = (\log(\abs{\sigma_1(x)}), 
	\dots, \log(\norm{\sigma_{r+s}(x)}^2))$ .The image $\im L$ is a lattice in 
	$H = ((1, \dots, 1)^t)^\perp$.

	\begin{Def}{Regulator}{}
		The \emph{regulator} $R_K$ of $K$ is 
		$$R_K = \frac{1}{\sqrt{r+s}} \cdot \covol((L(\OK^*) \subset H))$$
	\end{Def}

	Let $v = \frac{1}{\sqrt{r+s}} (1, \dots, 1)^t$, which is of norm $1$. 
	Then $R_K = \covol(\Lambda = (L(\eps_1), \dots, L(\eps_{r+s-1}), v))$


	\begin{prop}{}{}
		$R_K = \abs{\det (M_{ij})}$ with $(M_{ij}) = (L(\eps_1), \dots, v)$
	\end{prop}

	\begin{cor}{}{}
		$R_K = \abs{\det((L(\eps_1), \dots, L(\eps_{r+s-1})))}$
	\end{cor}

	\begin{ex}
		Let $K = \Q[\sqrt{d}]$ and $\eps = a+b\sqrt{d}$ be a generator of $U_K$.
		$R_K = \log\abs{z}$.
	\end{ex}

	\section{Dedekind zeta Function}

	\begin{Def}{}{}
	Let $K \vert \Q$ be a number field of degree $n = r + 2s$. 
	Let $\sigma_i, \dots, \sigma_{r+s}$ be the embeddings of $K$ in $\C$,
	$h := \Cl(\OK)$ and $\R_K$ the regulator of $K$. The \emph{Dirichlet zeta
	function} is defined by 
	$$\zeta_K (s) = \sum_{0 \neq \aa \subset \OK} \frac{1}{\Nm(\aa)^s}$$
	\end{Def}
	

	\begin{rem}
		For $K = \Q$ the function is well defined for $s > 1$ (even for 
		$s \in \C, \Re(s) > 1$, not relevant here though). Note that the function
		$x \mapsto x^{-s}$ is strictly decreasing. Thus $\int_{n-1}^{n} 
		x^{-s}\; \mathrm{d}x  > \frac{1}{n^s} > \int_{n}^{n+1} x^{-s}\; \mathrm{d}x  
		$. Hence $1+ \int_{1}^{\infty} x^{-s}\; \mathrm{d}x > \zeta_\Q(s) > 
		\int_{1}^{\infty} x^{-s} \; \mathrm{d}x$. Therefore $s > (s-1)\zeta_\Q(s) 
		> 1$. Thus 
		$$\lim_{s \to 1^+} \zeta_{\Q}(s)(s-1) = 1$$.
	\end{rem}

	\begin{thm}{Dream}{}
		$\lim_{s \to 1^+} \zeta_K(s)(s-1) = h_K \cdot c_K$ for some easy function 
		$c_K$ depending on $r, s, R_K$.
	\end{thm}

	\underline{$1^\circ$ trick}: 
	\begin{prop}{}{}
	$$\zeta_K(s) = \sum_{c \in \Cl(\OK)} 
	\sum_{[\aa] = c} \frac{1}{\Nm(\aa)^s} =: \sum_{c\in \Cl(\OK)} \zeta_c(s)$$.
	\end{prop}

	\begin{Def}{Fundamental domain}{}
		Let $l^* = (1,\dots,1,2,\dots,2)$ ($r$ and $s$ times respectively). 
		Then $l^*, L(\eps_1), \dots, L(\eps_{r+s-1})$ is a basis of $\R^{r+s}$
		for $\eps_i$ being a list of generators of $U_K$. Then the 
		\emph{fundamental doamin} $X \subset \R^n$ is defined by 
		\begin{align*}
			X = \{x \in \R^n \mid & 0 \leq \arg(x_1) \leq \frac{2\pi}{m}; \\
		 & \Nm(x) \neq 0; \\ 
		 & L(x) = \lambda l^* + \sum_{i=1}^{r+s-1} \lambda_i L(\eps_i)
		\forall \lambda \in \Rgez, \lambda_i \in [0,1) \}
		\end{align*}
	\end{Def}

	\begin{prop}{}{}
		Let $c \in \Cl(\OK)$ and $\aa'$ be a non-fractional ideal such that 
		$c^{-1} = [\aa']$. Then 
		\begin{align*}
			\zeta_C = \Norm(a')^s \sum_{(a) \subset \aa} \frac{1}{\abs{\Norm(a)}^s}
			= \Norm(\aa')^s \sum_{a \in \aa'; \sigma(a) \in X} \frac{1}{\abs{\Norm
			(a)}^s}
		\end{align*}
	\end{prop}

	Define 
	$S := \{x \in X \mid \Norm(x) = 1\} $ and 
	$T := \{x \in X \mid \Norm(x) \leq 1\} $.

	\begin{Lemma}{}{}
		$S$ is bounded.
	\end{Lemma}

	\begin{Lemma}{}{}
		$T$ is bounded.
	\end{Lemma}

	\begin{Lemma}{}{}
		Let $u \in U_K$ and $m_u: \R^n \to \R^n$ given by 
		$(x_1, \dots, z_{r+s})^t \mapsto (\sigma_1(u)x_1, \dots, 
		\sigma_{r+s}(u)z_{r+s})^t$. Then $m_u$ is volume preserving.
	\end{Lemma}

	\newcommand{\Tall}{T_{\text{all}}}

	Let $\mu(K) = \gen{\eta}$. Define $T_K := \eta^k T$ for $k = 0, \dots, m-1$.
	Also define $T_{\text{all}} = \bigcup_{k=0}^{m-1} T_K$. Then clearly 
	$\mu(\Tall) = m \mu(T)$. Let $\bar{T} \subset \Tall$ such that 
	$x_i > 0$ for $i = 1, \dots, r$. Then 
	$$\mu(\bar{T}) = 2^{-r} \mu(\Tall) = \frac{m}{2^r}\mu(T)$$

	\begin{prop}{}{}
		$\vol(T) = \frac{R_K \cdot \pi^s \cdot 2^r}{m}$
	\end{prop}
	


	
	












\end{document}




